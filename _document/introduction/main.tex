\part{Introduction}

\chapter{Bối cảnh}

Thương mại điện tử (TMĐT) bao gồm tất cả các hoạt động kinh doanh qua phương tiện điện tử, từ marketing, bán hàng, phân phối đến thanh toán, gọi tắt là MSDP. Sử dụng các phương tiện như điện thoại, fax, truyền hình và mạng máy tính.

TMĐT đã phát triển vượt bậc, đặc biệt với vai trò chủ đạo của Internet. Sự mở rộng sang các thiết bị di động, hay còn gọi là M-commerce, đã làm cho TMĐT ngày càng đa dạng và phát triển hơn. Lĩnh vực này đã trải qua một hành trình phát triển ấn tượng, từ những giao dịch đơn giản qua email và website, đến sự bùng nổ của các nền tảng mua sắm trực tuyến tích hợp công nghệ thanh toán và vận chuyển hiện đại. 

Hiện nay, TMĐT không chỉ giới hạn trong mua bán hàng hóa mà còn mở rộng ra các dịch vụ số, nền tảng di động và mạng xã hội, mang lại trải nghiệm mua sắm toàn diện hơn. Một ví dụ cụ thể là mô hình TMĐT B2C (doanh nghiệp đến người tiêu dùng), kết hợp cả TMĐT truyền thống và di động, giúp doanh nghiệp tiết kiệm chi phí và người tiêu dùng mua sắm thuận tiện. TMĐT mang lại nhiều lợi ích như mở rộng lựa chọn sản phẩm, giá thấp, giao hàng nhanh và xây dựng cộng đồng thương mại điện tử phát triển, thay đổi cách doanh nghiệp và người tiêu dùng tương tác, tạo ra môi trường TMĐT ngày càng sôi động và hiệu quả hơn.

\chapter{Mục tiêu và nhiệm vụ}

Từ bối cảnh phát triển mạnh mẽ của thương mại điện tử, việc phân tích dữ liệu trở nên cực kỳ quan trọng để hiểu rõ nhu cầu và trải nghiệm của khách hàng. Trong đồ án môn học này, nhóm em tập trung vào việc \textbf{phân tích các yếu tố khách hàng quan tâm về sản phẩm điện tử trên nền tảng thương mại điện tử}. Thông qua việc sử dụng các công cụ và phương pháp phân tích dữ liệu hiện đại, chúng em sẽ khám phá ra những điểm cần chú trọng trong tập dữ liệu, điểm mạnh và điểm yếu của các sản phẩm điện tử được bán trên \textbf{Lazada và Tiki}. Kết quả phân tích sẽ cung cấp cái nhìn sâu sắc về hành vi và cảm xúc, sự hài lòng của khách hàng và những yếu tố ảnh hưởng đến quyết định mua hàng, từ đó đưa ra các đề xuất cải tiến chất lượng sản phẩm và dịch vụ trên nền tảng này.

Nhiệm vụ của bài báo cáo này là trình bày chi tiết quá trình phân tích dữ liệu trong lĩnh vực thương mại điện tử trên nền tảng Lazada, tập trung vào sản phẩm điện thoại và máy tính bảng. Các nhiệm vụ cụ thể bao gồm:

\begin{enumerate}
    \item Thu thập bộ dữ liệu từ sàn thương mại điện tử Lazada: bộ dữ liệu thứ nhất chứa thông tin về các sản phẩm điện thoại và máy tính bảng, bộ dữ liệu thứ hai chứa các đánh giá của khách hàng liên quan đến những sản phẩm này.
    \item Tiến hành lọc và làm sạch dữ liệu để đảm bảo tính chính xác và nhất quán.
    \item Thực hiện phân tích dữ liệu: đề xuất các giả thuyết và áp dụng các phương pháp phân tích trực quan, thống kê để đưa ra nhận định và đánh giá về tập dữ liệu. 
    \item Soạn thảo và trình bày báo cáo: mỗi thành viên trong nhóm sẽ viết phần báo cáo về các giả thuyết và phương pháp phân tích mà mình đảm nhận, sau đó tổng hợp lại thành một báo cáo hoàn chỉnh. Cuối cùng, nhóm sẽ trình bày nội dung này trước lớp.
    \item Định hướng tương lai: chúng em sẽ tiếp tục sử dụng bộ dữ liệu này để phân tích cho các nội dung dự án cuối kỳ, nhằm nâng cao chất lượng và độ sâu của nghiên cứu.
\end{enumerate}

Nhờ những phân tích và phương pháp tiếp cận này, báo cáo sẽ cung cấp cái nhìn toàn diện và sâu sắc về sự hài lòng của khách hàng và hiệu quả kinh doanh của sản phẩm điện tử trên Lazada, góp phần đưa ra những đề xuất cải thiện chất lượng sản phẩm và dịch vụ.


% \cite{jaderberg2014synthetic}


% \begin{center}
%     \begin{figure}[htb] % Các option h: here (đặt tại đây), p: page (đặt ở top next page), t: top, b:bottom
%         \centering
%         \includegraphics[scale = 0.1]{images/Logo chinh.png}
%         \caption{Caption}
%         \label{fig:enter-label}
%     \end{figure}
% \end{center}
