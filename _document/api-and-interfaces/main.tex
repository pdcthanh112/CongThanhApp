\part{API and Interfaces}

\section{Kết luận}
Trong quá trình thu thập và phân tích dữ liệu cho báo cáo này, nhóm đã rút ra một số kết luận quan trọng như sau:

\begin{itemize}
    \item \textbf{Về dữ liệu thu thập được}: Dữ liệu chưa hoàn chỉnh do bị ảnh hưởng bởi thông tin của người dùng, chẳng hạn như đề xuất sản phẩm tại cùng vị trí địa lý của người dùng và các sản phẩm có rating cao trên Lazada.
    \item \textbf{Về tập dữ liệu ‘Product’}: Dữ liệu cho thấy sự ảnh hưởng rõ rệt của vị trí người bán và tỷ lệ giảm giá đến số lượt bán ra. Những sản phẩm từ các vị trí bán hàng thuận lợi và có tỷ lệ giảm giá hấp dẫn thường có số lượt mua cao hơn.
    \item \textbf{Về tập dữ liệu ‘Review’}: Các đánh giá tích cực thường tập trung vào chất lượng sản phẩm, cho thấy người tiêu dùng đánh giá cao các sản phẩm có chất lượng tốt. Ngược lại, các đánh giá tiêu cực thường liên quan đến uy tín người bán, cho thấy sự quan tâm của người tiêu dùng đối với độ tin cậy và chất lượng dịch vụ từ người bán hàng. 
\end{itemize}

Từ những kết luận trên, nhóm nhận thấy tầm quan trọng của việc cải thiện chất lượng sản phẩm và dịch vụ bán hàng để nâng cao sự hài lòng và tăng cường lòng tin của khách hàng.



\section{Hướng phát triển}
Để nâng cao chất lượng và độ chính xác của báo cáo phân tích dữ liệu, nhóm đề xuất các hướng phát triển sau:

\begin{itemize}
    \item \textbf{Về dữ liệu thu thập}: Thu thập dữ liệu đầy đủ hơn để tránh sự ảnh hưởng của việc đề xuất sản phẩm cho người dùng. Tiếp tục thu thập dữ liệu trên sàn thương mại điện tử Tiki.
    \item \textbf{Về tập dữ liệu ‘Product’}: Mở rộng phân tích trên nhiều yếu tố ảnh hưởng đến lượt bán hơn như điểm đánh giá, tên nhãn hiệu, và các thuộc tính khác của sản phẩm.
    \item \textbf{Về tập dữ liệu ‘Review’}: Phân tích sâu hơn vào các yếu tố mà khách hàng quan tâm trong đánh giá, như chất lượng sản phẩm, đặc điểm kỹ thuật, và ngoại hình sản phẩm.
    \item \textbf{Phát triển UI/UX}: Xây dựng giao diện người dùng (GUI) để thuận tiện cho việc trực quan hóa dữ liệu. Phát triển hệ thống tự động, thêm vào chatbot tích hợp AI để có thể tương tác với người dùng.
\end{itemize}
Bằng cách thực hiện những hướng phát triển này, kết quả phân tích dữ liệu của nhóm sẽ trở nên toàn diện hơn, cung cấp những thông tin chi tiết và chính xác hơn để hỗ trợ quyết định chiến lược trong kinh doanh.