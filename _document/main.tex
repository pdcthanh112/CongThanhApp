\documentclass[oneside,a4paper,12pt]{extreport}

% Font tiếng Việt
%\usepackage[T5]{fontenc}
\usepackage[utf8]{vietnam}
\usepackage{tabularx}
%\DeclareTextSymbolDefault{\DH}{T1}

% Tài liệu tham khảo
\usepackage[
	sorting=nty,
	backend=bibtex,
	defernumbers=true]{biblatex}
\usepackage[unicode]{hyperref} % Bookmark tiếng Việt
\addbibresource{references.bib}

\makeatletter
\def\blx@maxline{80}
\makeatother

\usepackage{etoolbox}

\makeatletter
\@addtoreset{chapter}{part}
\makeatother

\pretocmd{\part}{\setcounter{chapter}{1}}{}{}


% Chèn hình, các hình trong luận văn được để trong thư mục Images/
\usepackage{graphicx}
\graphicspath{ {images/} }
\usepackage{amsfonts}
\usepackage{amssymb}
\usepackage{subfigure}

% Chèn và định dạng mã nguồn
\usepackage{listings}
\usepackage{color}
\definecolor{codegreen}{rgb}{0,0.6,0}
\definecolor{codegray}{rgb}{0.5,0.5,0.5}
\definecolor{codepurple}{rgb}{0.58,0,0.82}
\definecolor{backcolour}{rgb}{0.95,0.95,0.92}
\lstdefinestyle{mystyle}{
    backgroundcolor=\color{backcolour},   
    commentstyle=\color{codegreen},
    keywordstyle=\color{magenta},
    numberstyle=\tiny\color{codegray},
    stringstyle=\color{codepurple},
    basicstyle=\footnotesize,
    breakatwhitespace=false,         
    breaklines=true,                 
    captionpos=b,                    
    keepspaces=true,                 
    numbers=left,                    
    numbersep=5pt,                  
    showspaces=false,                
    showstringspaces=false,
    showtabs=false,                  
    tabsize=2
}
\lstset{style=mystyle}

% Chèn và định dạng mã giả
\usepackage{amsmath}
\usepackage{algorithm}
\usepackage[noend]{algpseudocode}
\makeatletter
\def\BState{\State\hskip-\ALG@thistlm}
\makeatother

% Bảng biểu
\usepackage{multirow}
\usepackage{array}
\newcolumntype{L}[1]{>{\raggedright\let\newline\\\arraybackslash\hspace{0pt}}m{#1}}
\newcolumntype{C}[1]{>{\centering\let\newline\\\arraybackslash\hspace{0pt}}m{#1}}
\newcolumntype{R}[1]{>{\raggedleft\let\newline\\\arraybackslash\hspace{0pt}}m{#1}}

% Đổi tên mặc định
% \renewcommand{\chaptername}{Chương}
% \renewcommand{\figurename}{Hình}
% \renewcommand{\tablename}{Bảng}
% \renewcommand{\contentsname}{Mục lục}
% \renewcommand{\listfigurename}{Danh sách hình}
% \renewcommand{\listtablename}{Danh sách bảng}
% \renewcommand{\appendixname}{Phụ lục}

\usepackage[Glenn]{fncychap}

\usepackage{titlesec}

\titleformat{\part}[display]
  {\normalfont\Large\bfseries\color{red}} % Định dạng: cỡ chữ, đậm, màu đỏ
  {\thepart.} % Số La Mã tự động với dấu chấm
  {1em} % Khoảng cách giữa số và tiêu đề
  {}


% Dãn dòng 1.5
\usepackage{setspace}
\onehalfspacing

% Thụt vào đầu dòng
% \usepackage{indentfirst}           %tao cmt cái này lại

% Canh lề
\usepackage[
  top=30mm,
  bottom=25mm,
  left=30mm,
  right=20mm,
  includefoot]{geometry}
  
% Trang bìa
\usepackage{tikz}
\usetikzlibrary{calc}
\newcommand\HRule{\rule{\textwidth}{1pt}}

\newcommand{\tenSV}{Nguyễn~Văn~A~-~Trần~Văn~B} % Dấu ~ là khoảng trắng không được tách (các chữ nối với nhau bằng dấu ~ sẽ nằm cùng 1 dòng
\newcommand{\mssv}{1234567}
\newcommand{\tenKL}{Sử~dụng~LaTeX trong Khoá~luận~tốt~nghiệp} % Chú ý dấu ~ trong tên khóa luận
\newcommand{\tenGVHD}{Tên~Giáo~Viên}
\newcommand{\tenBM}{Công nghệ tri thức}

\begin{document}

\part{Project management}

\part{Project management}

\part{Project management}

\input{project-management/management-approach/main.tex}





\chapter{Tools \& Infrastructures}
\begin{table}[ht]
    \centering
    \begin{tabularx}{\textwidth}{|l|X|}
        \hline
        \textbf{Category} & \textbf{Tools / Infrastructure} \\ \hline
        Programing language & Java, TypeScript, Dart, Python, Golang \\ \hline
        Framework & NodeJS (Expresss), Spring boot, ReactJS, Angular, Flutter, PyToch, Streamlit\\ \hline
        Database & PostgreSQL, MySQL, MongoDB \\ \hline
        IDEs/Editors & NeoVim, Visual Studio Code, IntelliJ, PyCharm, GoLand \\ \hline
        Diagramming & DrawIO, StarUML \\ \hline
        Documentation & Latex (Overfeaf) \\ \hline
        Version Control & GitHub (Source code), Google Drive (Documents) \\ \hline
        Deployment server & DigitalOcean, Azure, Netlify \\ \hline
       
        Algorithm & - Screening algorithm\n- Queue planning to send emails \\ \hline
    \end{tabularx}
    \caption{Tools and Infrastructure}
    \label{tab:tools-infrastructure}
\end{table}





\chapter{Tools \& Infrastructures}
\begin{table}[ht]
    \centering
    \begin{tabularx}{\textwidth}{|l|X|}
        \hline
        \textbf{Category} & \textbf{Tools / Infrastructure} \\ \hline
        Programing language & Java, TypeScript, Dart, Python, Golang \\ \hline
        Framework & NodeJS (Expresss), Spring boot, ReactJS, Angular, Flutter, PyToch, Streamlit\\ \hline
        Database & PostgreSQL, MySQL, MongoDB \\ \hline
        IDEs/Editors & NeoVim, Visual Studio Code, IntelliJ, PyCharm, GoLand \\ \hline
        Diagramming & DrawIO, StarUML \\ \hline
        Documentation & Latex (Overfeaf) \\ \hline
        Version Control & GitHub (Source code), Google Drive (Documents) \\ \hline
        Deployment server & DigitalOcean, Azure, Netlify \\ \hline
       
        Algorithm & - Screening algorithm\n- Queue planning to send emails \\ \hline
    \end{tabularx}
    \caption{Tools and Infrastructure}
    \label{tab:tools-infrastructure}
\end{table}





\chapter{Tools \& Infrastructures}
\begin{table}[ht]
    \centering
    \begin{tabularx}{\textwidth}{|l|X|}
        \hline
        \textbf{Category} & \textbf{Tools / Infrastructure} \\ \hline
        Programing language & Java, TypeScript, Dart, Python, Golang \\ \hline
        Framework & NodeJS (Expresss), Spring boot, ReactJS, Angular, Flutter, PyToch, Streamlit\\ \hline
        Database & PostgreSQL, MySQL, MongoDB \\ \hline
        IDEs/Editors & NeoVim, Visual Studio Code, IntelliJ, PyCharm, GoLand \\ \hline
        Diagramming & DrawIO, StarUML \\ \hline
        Documentation & Latex (Overfeaf) \\ \hline
        Version Control & GitHub (Source code), Google Drive (Documents) \\ \hline
        Deployment server & DigitalOcean, Azure, Netlify \\ \hline
       
        Algorithm & - Screening algorithm\n- Queue planning to send emails \\ \hline
    \end{tabularx}
    \caption{Tools and Infrastructure}
    \label{tab:tools-infrastructure}
\end{table}
\part{Project management}

\part{Project management}

\part{Project management}

\input{project-management/management-approach/main.tex}





\chapter{Tools \& Infrastructures}
\begin{table}[ht]
    \centering
    \begin{tabularx}{\textwidth}{|l|X|}
        \hline
        \textbf{Category} & \textbf{Tools / Infrastructure} \\ \hline
        Programing language & Java, TypeScript, Dart, Python, Golang \\ \hline
        Framework & NodeJS (Expresss), Spring boot, ReactJS, Angular, Flutter, PyToch, Streamlit\\ \hline
        Database & PostgreSQL, MySQL, MongoDB \\ \hline
        IDEs/Editors & NeoVim, Visual Studio Code, IntelliJ, PyCharm, GoLand \\ \hline
        Diagramming & DrawIO, StarUML \\ \hline
        Documentation & Latex (Overfeaf) \\ \hline
        Version Control & GitHub (Source code), Google Drive (Documents) \\ \hline
        Deployment server & DigitalOcean, Azure, Netlify \\ \hline
       
        Algorithm & - Screening algorithm\n- Queue planning to send emails \\ \hline
    \end{tabularx}
    \caption{Tools and Infrastructure}
    \label{tab:tools-infrastructure}
\end{table}





\chapter{Tools \& Infrastructures}
\begin{table}[ht]
    \centering
    \begin{tabularx}{\textwidth}{|l|X|}
        \hline
        \textbf{Category} & \textbf{Tools / Infrastructure} \\ \hline
        Programing language & Java, TypeScript, Dart, Python, Golang \\ \hline
        Framework & NodeJS (Expresss), Spring boot, ReactJS, Angular, Flutter, PyToch, Streamlit\\ \hline
        Database & PostgreSQL, MySQL, MongoDB \\ \hline
        IDEs/Editors & NeoVim, Visual Studio Code, IntelliJ, PyCharm, GoLand \\ \hline
        Diagramming & DrawIO, StarUML \\ \hline
        Documentation & Latex (Overfeaf) \\ \hline
        Version Control & GitHub (Source code), Google Drive (Documents) \\ \hline
        Deployment server & DigitalOcean, Azure, Netlify \\ \hline
       
        Algorithm & - Screening algorithm\n- Queue planning to send emails \\ \hline
    \end{tabularx}
    \caption{Tools and Infrastructure}
    \label{tab:tools-infrastructure}
\end{table}





\chapter{Tools \& Infrastructures}
\begin{table}[ht]
    \centering
    \begin{tabularx}{\textwidth}{|l|X|}
        \hline
        \textbf{Category} & \textbf{Tools / Infrastructure} \\ \hline
        Programing language & Java, TypeScript, Dart, Python, Golang \\ \hline
        Framework & NodeJS (Expresss), Spring boot, ReactJS, Angular, Flutter, PyToch, Streamlit\\ \hline
        Database & PostgreSQL, MySQL, MongoDB \\ \hline
        IDEs/Editors & NeoVim, Visual Studio Code, IntelliJ, PyCharm, GoLand \\ \hline
        Diagramming & DrawIO, StarUML \\ \hline
        Documentation & Latex (Overfeaf) \\ \hline
        Version Control & GitHub (Source code), Google Drive (Documents) \\ \hline
        Deployment server & DigitalOcean, Azure, Netlify \\ \hline
       
        Algorithm & - Screening algorithm\n- Queue planning to send emails \\ \hline
    \end{tabularx}
    \caption{Tools and Infrastructure}
    \label{tab:tools-infrastructure}
\end{table}

\pagenumbering{roman} % Đánh số i, ii, iii, ...


% Mục lục, danh sách hình, danh sách bảng
%\addcontentsline{toc}{chapter}{Mục lục}
\tableofcontents
\listoffigures
\listoftables

\titleformat{\chapter}
  {\normalfont\Large\bfseries\color{blue}} % Định dạng: cỡ chữ, đậm, màu đỏ
  {\thechapter.}
  {1em} % Khoảng cách giữa số và tiêu đề
  {}

%\addcontentsline{toc}{chapter}{Tóm tắt}
\paragraph{\Large Acknowledgements\\[0.5cm]}
{I am deeply honored and humbled to have had the privilege of pursuing my doctoral degree at the prestigious Massachusetts Institute of Technology (MIT). The unwavering support, invaluable guidance, and exceptional mentorship of the esteemed faculty have been instrumental in shaping the trajectory and success of this research.

At the forefront of my gratitude stands Professor Dr. Andrew Ng, whose visionary leadership and unparalleled expertise have been a constant source of inspiration throughout my academic journey. His insightful feedback, unwavering encouragement, and commitment to propelling the boundaries of knowledge have been truly transformative. Under his tutelage, I have not only honed my technical prowess but have also developed a profound appreciation for the power of innovation and the pursuit of groundbreaking discoveries.

The vibrant and collaborative environment fostered by the MIT community has been instrumental in fueling my intellectual growth and fostering a deep sense of camaraderie. The university's renowned laboratories, cutting-edge resources, and world-class facilities have provided an unparalleled platform for my research endeavors, enabling me to push the limits of what is possible in the field of computer science.

I am immensely grateful to the United States government for the generous doctoral fellowship that has made this journey possible. The financial support and access to research opportunities have been pivotal in allowing me to delve deeply into the complexities of my chosen field and contribute meaningfully to the advancement of knowledge.

Furthermore, I extend my heartfelt appreciation to the tech giants Google and Meta for their invaluable sponsorship and collaborative support. Their willingness to share their state-of-the-art technologies, computing infrastructure, and technical expertise has been instrumental in shaping the direction and depth of my research. The synergistic partnership between academia and industry has been a driving force in propelling my work forward and ensuring its real-world relevance and impact.

As I reflect on this remarkable chapter of my life, I am filled with a deep sense of gratitude and a renewed commitment to leveraging the knowledge and skills I have acquired to make a meaningful difference in the world. I am truly honored to have been entrusted with this opportunity, and I pledge to continue pushing the boundaries of innovation and discovery, inspired by the unwavering support and guidance of this esteemed community.
}


\hfill
\begin{minipage}[t]{0.5\textwidth}\raggedright
Best regards,\\
Thanh Pham
\end{minipage}


Definition and Acronyms

\begin{table}[htbp]
    \centering
    \begin{tabularx}{\textwidth}{|c|X|}
        \hline
        \textbf{Acronym} & \textbf{Definition} \\ \hline
        BR & Business Rule\\ \hline
        ERD & Entity Relationship Diagram\\ \hline
        SRS & Software Requirement Specification\\ \hline
        API & Application Program Interface\\ \hline
        UC & Use Case\\ \hline
        UAT & User Acceptance Test\\ \hline
    \end{tabularx}
    \caption{Definition and Acronyms}
    \label{tab:Definition-and-Acronyms}
\end{table}


\clearpage

\pagenumbering{arabic} % Đánh số 1, 2, 3, ...

% Các chương nội dung
\part{Project management}

\part{Project management}

\part{Project management}

\input{project-management/management-approach/main.tex}





\chapter{Tools \& Infrastructures}
\begin{table}[ht]
    \centering
    \begin{tabularx}{\textwidth}{|l|X|}
        \hline
        \textbf{Category} & \textbf{Tools / Infrastructure} \\ \hline
        Programing language & Java, TypeScript, Dart, Python, Golang \\ \hline
        Framework & NodeJS (Expresss), Spring boot, ReactJS, Angular, Flutter, PyToch, Streamlit\\ \hline
        Database & PostgreSQL, MySQL, MongoDB \\ \hline
        IDEs/Editors & NeoVim, Visual Studio Code, IntelliJ, PyCharm, GoLand \\ \hline
        Diagramming & DrawIO, StarUML \\ \hline
        Documentation & Latex (Overfeaf) \\ \hline
        Version Control & GitHub (Source code), Google Drive (Documents) \\ \hline
        Deployment server & DigitalOcean, Azure, Netlify \\ \hline
       
        Algorithm & - Screening algorithm\n- Queue planning to send emails \\ \hline
    \end{tabularx}
    \caption{Tools and Infrastructure}
    \label{tab:tools-infrastructure}
\end{table}





\chapter{Tools \& Infrastructures}
\begin{table}[ht]
    \centering
    \begin{tabularx}{\textwidth}{|l|X|}
        \hline
        \textbf{Category} & \textbf{Tools / Infrastructure} \\ \hline
        Programing language & Java, TypeScript, Dart, Python, Golang \\ \hline
        Framework & NodeJS (Expresss), Spring boot, ReactJS, Angular, Flutter, PyToch, Streamlit\\ \hline
        Database & PostgreSQL, MySQL, MongoDB \\ \hline
        IDEs/Editors & NeoVim, Visual Studio Code, IntelliJ, PyCharm, GoLand \\ \hline
        Diagramming & DrawIO, StarUML \\ \hline
        Documentation & Latex (Overfeaf) \\ \hline
        Version Control & GitHub (Source code), Google Drive (Documents) \\ \hline
        Deployment server & DigitalOcean, Azure, Netlify \\ \hline
       
        Algorithm & - Screening algorithm\n- Queue planning to send emails \\ \hline
    \end{tabularx}
    \caption{Tools and Infrastructure}
    \label{tab:tools-infrastructure}
\end{table}





\chapter{Tools \& Infrastructures}
\begin{table}[ht]
    \centering
    \begin{tabularx}{\textwidth}{|l|X|}
        \hline
        \textbf{Category} & \textbf{Tools / Infrastructure} \\ \hline
        Programing language & Java, TypeScript, Dart, Python, Golang \\ \hline
        Framework & NodeJS (Expresss), Spring boot, ReactJS, Angular, Flutter, PyToch, Streamlit\\ \hline
        Database & PostgreSQL, MySQL, MongoDB \\ \hline
        IDEs/Editors & NeoVim, Visual Studio Code, IntelliJ, PyCharm, GoLand \\ \hline
        Diagramming & DrawIO, StarUML \\ \hline
        Documentation & Latex (Overfeaf) \\ \hline
        Version Control & GitHub (Source code), Google Drive (Documents) \\ \hline
        Deployment server & DigitalOcean, Azure, Netlify \\ \hline
       
        Algorithm & - Screening algorithm\n- Queue planning to send emails \\ \hline
    \end{tabularx}
    \caption{Tools and Infrastructure}
    \label{tab:tools-infrastructure}
\end{table}
\part{Project management}

\part{Project management}

\part{Project management}

\input{project-management/management-approach/main.tex}





\chapter{Tools \& Infrastructures}
\begin{table}[ht]
    \centering
    \begin{tabularx}{\textwidth}{|l|X|}
        \hline
        \textbf{Category} & \textbf{Tools / Infrastructure} \\ \hline
        Programing language & Java, TypeScript, Dart, Python, Golang \\ \hline
        Framework & NodeJS (Expresss), Spring boot, ReactJS, Angular, Flutter, PyToch, Streamlit\\ \hline
        Database & PostgreSQL, MySQL, MongoDB \\ \hline
        IDEs/Editors & NeoVim, Visual Studio Code, IntelliJ, PyCharm, GoLand \\ \hline
        Diagramming & DrawIO, StarUML \\ \hline
        Documentation & Latex (Overfeaf) \\ \hline
        Version Control & GitHub (Source code), Google Drive (Documents) \\ \hline
        Deployment server & DigitalOcean, Azure, Netlify \\ \hline
       
        Algorithm & - Screening algorithm\n- Queue planning to send emails \\ \hline
    \end{tabularx}
    \caption{Tools and Infrastructure}
    \label{tab:tools-infrastructure}
\end{table}





\chapter{Tools \& Infrastructures}
\begin{table}[ht]
    \centering
    \begin{tabularx}{\textwidth}{|l|X|}
        \hline
        \textbf{Category} & \textbf{Tools / Infrastructure} \\ \hline
        Programing language & Java, TypeScript, Dart, Python, Golang \\ \hline
        Framework & NodeJS (Expresss), Spring boot, ReactJS, Angular, Flutter, PyToch, Streamlit\\ \hline
        Database & PostgreSQL, MySQL, MongoDB \\ \hline
        IDEs/Editors & NeoVim, Visual Studio Code, IntelliJ, PyCharm, GoLand \\ \hline
        Diagramming & DrawIO, StarUML \\ \hline
        Documentation & Latex (Overfeaf) \\ \hline
        Version Control & GitHub (Source code), Google Drive (Documents) \\ \hline
        Deployment server & DigitalOcean, Azure, Netlify \\ \hline
       
        Algorithm & - Screening algorithm\n- Queue planning to send emails \\ \hline
    \end{tabularx}
    \caption{Tools and Infrastructure}
    \label{tab:tools-infrastructure}
\end{table}





\chapter{Tools \& Infrastructures}
\begin{table}[ht]
    \centering
    \begin{tabularx}{\textwidth}{|l|X|}
        \hline
        \textbf{Category} & \textbf{Tools / Infrastructure} \\ \hline
        Programing language & Java, TypeScript, Dart, Python, Golang \\ \hline
        Framework & NodeJS (Expresss), Spring boot, ReactJS, Angular, Flutter, PyToch, Streamlit\\ \hline
        Database & PostgreSQL, MySQL, MongoDB \\ \hline
        IDEs/Editors & NeoVim, Visual Studio Code, IntelliJ, PyCharm, GoLand \\ \hline
        Diagramming & DrawIO, StarUML \\ \hline
        Documentation & Latex (Overfeaf) \\ \hline
        Version Control & GitHub (Source code), Google Drive (Documents) \\ \hline
        Deployment server & DigitalOcean, Azure, Netlify \\ \hline
       
        Algorithm & - Screening algorithm\n- Queue planning to send emails \\ \hline
    \end{tabularx}
    \caption{Tools and Infrastructure}
    \label{tab:tools-infrastructure}
\end{table}
\part{Project management}

\part{Project management}

\part{Project management}

\input{project-management/management-approach/main.tex}





\chapter{Tools \& Infrastructures}
\begin{table}[ht]
    \centering
    \begin{tabularx}{\textwidth}{|l|X|}
        \hline
        \textbf{Category} & \textbf{Tools / Infrastructure} \\ \hline
        Programing language & Java, TypeScript, Dart, Python, Golang \\ \hline
        Framework & NodeJS (Expresss), Spring boot, ReactJS, Angular, Flutter, PyToch, Streamlit\\ \hline
        Database & PostgreSQL, MySQL, MongoDB \\ \hline
        IDEs/Editors & NeoVim, Visual Studio Code, IntelliJ, PyCharm, GoLand \\ \hline
        Diagramming & DrawIO, StarUML \\ \hline
        Documentation & Latex (Overfeaf) \\ \hline
        Version Control & GitHub (Source code), Google Drive (Documents) \\ \hline
        Deployment server & DigitalOcean, Azure, Netlify \\ \hline
       
        Algorithm & - Screening algorithm\n- Queue planning to send emails \\ \hline
    \end{tabularx}
    \caption{Tools and Infrastructure}
    \label{tab:tools-infrastructure}
\end{table}





\chapter{Tools \& Infrastructures}
\begin{table}[ht]
    \centering
    \begin{tabularx}{\textwidth}{|l|X|}
        \hline
        \textbf{Category} & \textbf{Tools / Infrastructure} \\ \hline
        Programing language & Java, TypeScript, Dart, Python, Golang \\ \hline
        Framework & NodeJS (Expresss), Spring boot, ReactJS, Angular, Flutter, PyToch, Streamlit\\ \hline
        Database & PostgreSQL, MySQL, MongoDB \\ \hline
        IDEs/Editors & NeoVim, Visual Studio Code, IntelliJ, PyCharm, GoLand \\ \hline
        Diagramming & DrawIO, StarUML \\ \hline
        Documentation & Latex (Overfeaf) \\ \hline
        Version Control & GitHub (Source code), Google Drive (Documents) \\ \hline
        Deployment server & DigitalOcean, Azure, Netlify \\ \hline
       
        Algorithm & - Screening algorithm\n- Queue planning to send emails \\ \hline
    \end{tabularx}
    \caption{Tools and Infrastructure}
    \label{tab:tools-infrastructure}
\end{table}





\chapter{Tools \& Infrastructures}
\begin{table}[ht]
    \centering
    \begin{tabularx}{\textwidth}{|l|X|}
        \hline
        \textbf{Category} & \textbf{Tools / Infrastructure} \\ \hline
        Programing language & Java, TypeScript, Dart, Python, Golang \\ \hline
        Framework & NodeJS (Expresss), Spring boot, ReactJS, Angular, Flutter, PyToch, Streamlit\\ \hline
        Database & PostgreSQL, MySQL, MongoDB \\ \hline
        IDEs/Editors & NeoVim, Visual Studio Code, IntelliJ, PyCharm, GoLand \\ \hline
        Diagramming & DrawIO, StarUML \\ \hline
        Documentation & Latex (Overfeaf) \\ \hline
        Version Control & GitHub (Source code), Google Drive (Documents) \\ \hline
        Deployment server & DigitalOcean, Azure, Netlify \\ \hline
       
        Algorithm & - Screening algorithm\n- Queue planning to send emails \\ \hline
    \end{tabularx}
    \caption{Tools and Infrastructure}
    \label{tab:tools-infrastructure}
\end{table}
\part{Project management}

\part{Project management}

\part{Project management}

\input{project-management/management-approach/main.tex}





\chapter{Tools \& Infrastructures}
\begin{table}[ht]
    \centering
    \begin{tabularx}{\textwidth}{|l|X|}
        \hline
        \textbf{Category} & \textbf{Tools / Infrastructure} \\ \hline
        Programing language & Java, TypeScript, Dart, Python, Golang \\ \hline
        Framework & NodeJS (Expresss), Spring boot, ReactJS, Angular, Flutter, PyToch, Streamlit\\ \hline
        Database & PostgreSQL, MySQL, MongoDB \\ \hline
        IDEs/Editors & NeoVim, Visual Studio Code, IntelliJ, PyCharm, GoLand \\ \hline
        Diagramming & DrawIO, StarUML \\ \hline
        Documentation & Latex (Overfeaf) \\ \hline
        Version Control & GitHub (Source code), Google Drive (Documents) \\ \hline
        Deployment server & DigitalOcean, Azure, Netlify \\ \hline
       
        Algorithm & - Screening algorithm\n- Queue planning to send emails \\ \hline
    \end{tabularx}
    \caption{Tools and Infrastructure}
    \label{tab:tools-infrastructure}
\end{table}





\chapter{Tools \& Infrastructures}
\begin{table}[ht]
    \centering
    \begin{tabularx}{\textwidth}{|l|X|}
        \hline
        \textbf{Category} & \textbf{Tools / Infrastructure} \\ \hline
        Programing language & Java, TypeScript, Dart, Python, Golang \\ \hline
        Framework & NodeJS (Expresss), Spring boot, ReactJS, Angular, Flutter, PyToch, Streamlit\\ \hline
        Database & PostgreSQL, MySQL, MongoDB \\ \hline
        IDEs/Editors & NeoVim, Visual Studio Code, IntelliJ, PyCharm, GoLand \\ \hline
        Diagramming & DrawIO, StarUML \\ \hline
        Documentation & Latex (Overfeaf) \\ \hline
        Version Control & GitHub (Source code), Google Drive (Documents) \\ \hline
        Deployment server & DigitalOcean, Azure, Netlify \\ \hline
       
        Algorithm & - Screening algorithm\n- Queue planning to send emails \\ \hline
    \end{tabularx}
    \caption{Tools and Infrastructure}
    \label{tab:tools-infrastructure}
\end{table}





\chapter{Tools \& Infrastructures}
\begin{table}[ht]
    \centering
    \begin{tabularx}{\textwidth}{|l|X|}
        \hline
        \textbf{Category} & \textbf{Tools / Infrastructure} \\ \hline
        Programing language & Java, TypeScript, Dart, Python, Golang \\ \hline
        Framework & NodeJS (Expresss), Spring boot, ReactJS, Angular, Flutter, PyToch, Streamlit\\ \hline
        Database & PostgreSQL, MySQL, MongoDB \\ \hline
        IDEs/Editors & NeoVim, Visual Studio Code, IntelliJ, PyCharm, GoLand \\ \hline
        Diagramming & DrawIO, StarUML \\ \hline
        Documentation & Latex (Overfeaf) \\ \hline
        Version Control & GitHub (Source code), Google Drive (Documents) \\ \hline
        Deployment server & DigitalOcean, Azure, Netlify \\ \hline
       
        Algorithm & - Screening algorithm\n- Queue planning to send emails \\ \hline
    \end{tabularx}
    \caption{Tools and Infrastructure}
    \label{tab:tools-infrastructure}
\end{table}
\part{Project management}

\part{Project management}

\part{Project management}

\input{project-management/management-approach/main.tex}





\chapter{Tools \& Infrastructures}
\begin{table}[ht]
    \centering
    \begin{tabularx}{\textwidth}{|l|X|}
        \hline
        \textbf{Category} & \textbf{Tools / Infrastructure} \\ \hline
        Programing language & Java, TypeScript, Dart, Python, Golang \\ \hline
        Framework & NodeJS (Expresss), Spring boot, ReactJS, Angular, Flutter, PyToch, Streamlit\\ \hline
        Database & PostgreSQL, MySQL, MongoDB \\ \hline
        IDEs/Editors & NeoVim, Visual Studio Code, IntelliJ, PyCharm, GoLand \\ \hline
        Diagramming & DrawIO, StarUML \\ \hline
        Documentation & Latex (Overfeaf) \\ \hline
        Version Control & GitHub (Source code), Google Drive (Documents) \\ \hline
        Deployment server & DigitalOcean, Azure, Netlify \\ \hline
       
        Algorithm & - Screening algorithm\n- Queue planning to send emails \\ \hline
    \end{tabularx}
    \caption{Tools and Infrastructure}
    \label{tab:tools-infrastructure}
\end{table}





\chapter{Tools \& Infrastructures}
\begin{table}[ht]
    \centering
    \begin{tabularx}{\textwidth}{|l|X|}
        \hline
        \textbf{Category} & \textbf{Tools / Infrastructure} \\ \hline
        Programing language & Java, TypeScript, Dart, Python, Golang \\ \hline
        Framework & NodeJS (Expresss), Spring boot, ReactJS, Angular, Flutter, PyToch, Streamlit\\ \hline
        Database & PostgreSQL, MySQL, MongoDB \\ \hline
        IDEs/Editors & NeoVim, Visual Studio Code, IntelliJ, PyCharm, GoLand \\ \hline
        Diagramming & DrawIO, StarUML \\ \hline
        Documentation & Latex (Overfeaf) \\ \hline
        Version Control & GitHub (Source code), Google Drive (Documents) \\ \hline
        Deployment server & DigitalOcean, Azure, Netlify \\ \hline
       
        Algorithm & - Screening algorithm\n- Queue planning to send emails \\ \hline
    \end{tabularx}
    \caption{Tools and Infrastructure}
    \label{tab:tools-infrastructure}
\end{table}





\chapter{Tools \& Infrastructures}
\begin{table}[ht]
    \centering
    \begin{tabularx}{\textwidth}{|l|X|}
        \hline
        \textbf{Category} & \textbf{Tools / Infrastructure} \\ \hline
        Programing language & Java, TypeScript, Dart, Python, Golang \\ \hline
        Framework & NodeJS (Expresss), Spring boot, ReactJS, Angular, Flutter, PyToch, Streamlit\\ \hline
        Database & PostgreSQL, MySQL, MongoDB \\ \hline
        IDEs/Editors & NeoVim, Visual Studio Code, IntelliJ, PyCharm, GoLand \\ \hline
        Diagramming & DrawIO, StarUML \\ \hline
        Documentation & Latex (Overfeaf) \\ \hline
        Version Control & GitHub (Source code), Google Drive (Documents) \\ \hline
        Deployment server & DigitalOcean, Azure, Netlify \\ \hline
       
        Algorithm & - Screening algorithm\n- Queue planning to send emails \\ \hline
    \end{tabularx}
    \caption{Tools and Infrastructure}
    \label{tab:tools-infrastructure}
\end{table}
\part{Project management}

\part{Project management}

\part{Project management}

\input{project-management/management-approach/main.tex}





\chapter{Tools \& Infrastructures}
\begin{table}[ht]
    \centering
    \begin{tabularx}{\textwidth}{|l|X|}
        \hline
        \textbf{Category} & \textbf{Tools / Infrastructure} \\ \hline
        Programing language & Java, TypeScript, Dart, Python, Golang \\ \hline
        Framework & NodeJS (Expresss), Spring boot, ReactJS, Angular, Flutter, PyToch, Streamlit\\ \hline
        Database & PostgreSQL, MySQL, MongoDB \\ \hline
        IDEs/Editors & NeoVim, Visual Studio Code, IntelliJ, PyCharm, GoLand \\ \hline
        Diagramming & DrawIO, StarUML \\ \hline
        Documentation & Latex (Overfeaf) \\ \hline
        Version Control & GitHub (Source code), Google Drive (Documents) \\ \hline
        Deployment server & DigitalOcean, Azure, Netlify \\ \hline
       
        Algorithm & - Screening algorithm\n- Queue planning to send emails \\ \hline
    \end{tabularx}
    \caption{Tools and Infrastructure}
    \label{tab:tools-infrastructure}
\end{table}





\chapter{Tools \& Infrastructures}
\begin{table}[ht]
    \centering
    \begin{tabularx}{\textwidth}{|l|X|}
        \hline
        \textbf{Category} & \textbf{Tools / Infrastructure} \\ \hline
        Programing language & Java, TypeScript, Dart, Python, Golang \\ \hline
        Framework & NodeJS (Expresss), Spring boot, ReactJS, Angular, Flutter, PyToch, Streamlit\\ \hline
        Database & PostgreSQL, MySQL, MongoDB \\ \hline
        IDEs/Editors & NeoVim, Visual Studio Code, IntelliJ, PyCharm, GoLand \\ \hline
        Diagramming & DrawIO, StarUML \\ \hline
        Documentation & Latex (Overfeaf) \\ \hline
        Version Control & GitHub (Source code), Google Drive (Documents) \\ \hline
        Deployment server & DigitalOcean, Azure, Netlify \\ \hline
       
        Algorithm & - Screening algorithm\n- Queue planning to send emails \\ \hline
    \end{tabularx}
    \caption{Tools and Infrastructure}
    \label{tab:tools-infrastructure}
\end{table}





\chapter{Tools \& Infrastructures}
\begin{table}[ht]
    \centering
    \begin{tabularx}{\textwidth}{|l|X|}
        \hline
        \textbf{Category} & \textbf{Tools / Infrastructure} \\ \hline
        Programing language & Java, TypeScript, Dart, Python, Golang \\ \hline
        Framework & NodeJS (Expresss), Spring boot, ReactJS, Angular, Flutter, PyToch, Streamlit\\ \hline
        Database & PostgreSQL, MySQL, MongoDB \\ \hline
        IDEs/Editors & NeoVim, Visual Studio Code, IntelliJ, PyCharm, GoLand \\ \hline
        Diagramming & DrawIO, StarUML \\ \hline
        Documentation & Latex (Overfeaf) \\ \hline
        Version Control & GitHub (Source code), Google Drive (Documents) \\ \hline
        Deployment server & DigitalOcean, Azure, Netlify \\ \hline
       
        Algorithm & - Screening algorithm\n- Queue planning to send emails \\ \hline
    \end{tabularx}
    \caption{Tools and Infrastructure}
    \label{tab:tools-infrastructure}
\end{table}
\part{Project management}

\part{Project management}

\part{Project management}

\input{project-management/management-approach/main.tex}





\chapter{Tools \& Infrastructures}
\begin{table}[ht]
    \centering
    \begin{tabularx}{\textwidth}{|l|X|}
        \hline
        \textbf{Category} & \textbf{Tools / Infrastructure} \\ \hline
        Programing language & Java, TypeScript, Dart, Python, Golang \\ \hline
        Framework & NodeJS (Expresss), Spring boot, ReactJS, Angular, Flutter, PyToch, Streamlit\\ \hline
        Database & PostgreSQL, MySQL, MongoDB \\ \hline
        IDEs/Editors & NeoVim, Visual Studio Code, IntelliJ, PyCharm, GoLand \\ \hline
        Diagramming & DrawIO, StarUML \\ \hline
        Documentation & Latex (Overfeaf) \\ \hline
        Version Control & GitHub (Source code), Google Drive (Documents) \\ \hline
        Deployment server & DigitalOcean, Azure, Netlify \\ \hline
       
        Algorithm & - Screening algorithm\n- Queue planning to send emails \\ \hline
    \end{tabularx}
    \caption{Tools and Infrastructure}
    \label{tab:tools-infrastructure}
\end{table}





\chapter{Tools \& Infrastructures}
\begin{table}[ht]
    \centering
    \begin{tabularx}{\textwidth}{|l|X|}
        \hline
        \textbf{Category} & \textbf{Tools / Infrastructure} \\ \hline
        Programing language & Java, TypeScript, Dart, Python, Golang \\ \hline
        Framework & NodeJS (Expresss), Spring boot, ReactJS, Angular, Flutter, PyToch, Streamlit\\ \hline
        Database & PostgreSQL, MySQL, MongoDB \\ \hline
        IDEs/Editors & NeoVim, Visual Studio Code, IntelliJ, PyCharm, GoLand \\ \hline
        Diagramming & DrawIO, StarUML \\ \hline
        Documentation & Latex (Overfeaf) \\ \hline
        Version Control & GitHub (Source code), Google Drive (Documents) \\ \hline
        Deployment server & DigitalOcean, Azure, Netlify \\ \hline
       
        Algorithm & - Screening algorithm\n- Queue planning to send emails \\ \hline
    \end{tabularx}
    \caption{Tools and Infrastructure}
    \label{tab:tools-infrastructure}
\end{table}





\chapter{Tools \& Infrastructures}
\begin{table}[ht]
    \centering
    \begin{tabularx}{\textwidth}{|l|X|}
        \hline
        \textbf{Category} & \textbf{Tools / Infrastructure} \\ \hline
        Programing language & Java, TypeScript, Dart, Python, Golang \\ \hline
        Framework & NodeJS (Expresss), Spring boot, ReactJS, Angular, Flutter, PyToch, Streamlit\\ \hline
        Database & PostgreSQL, MySQL, MongoDB \\ \hline
        IDEs/Editors & NeoVim, Visual Studio Code, IntelliJ, PyCharm, GoLand \\ \hline
        Diagramming & DrawIO, StarUML \\ \hline
        Documentation & Latex (Overfeaf) \\ \hline
        Version Control & GitHub (Source code), Google Drive (Documents) \\ \hline
        Deployment server & DigitalOcean, Azure, Netlify \\ \hline
       
        Algorithm & - Screening algorithm\n- Queue planning to send emails \\ \hline
    \end{tabularx}
    \caption{Tools and Infrastructure}
    \label{tab:tools-infrastructure}
\end{table}


% In tài liệu tham khảo
\addcontentsline{toc}{chapter}{Tài liệu tham khảo}

\DeclareNameAlias{sortname}{last-first}
\DeclareNameAlias{default}{last-first}

\printbibliography[title={Tài liệu tham khảo}, notkeyword=Viet, resetnumbers] 
% ===================================================================== %
% CHÚ Ý: phải gán lại resetnumbers=số tài liệu tham khảo tiếng Việt + 1 %
% ===================================================================== %


\end{document} 